This dissertation shows that record-replay is useful beyond deterministic replay.

Application record and replay is the ability to record application execution and
replay it at a later time. Record-replay has many use cases including diagnosing and
debugging applications by capturing and reproducing hard to find bugs, providing
transparent application fault tolerance by maintaining a live replica of a running program,
and offline instrumentation that would be too costly to run in a production environment.
Different record-replay systems may offer different levels of replay faithfulness,
the strongest level being deterministic replay which guarantees an identical
reenactment of the original execution. However, such deterministic record-replay systems
can dramatically hinder application performance, rendering them unpractical in
certain application domains. Further, other use cases are incompatible with
deterministic record-replay. For example, in a primary-secondary database scenario,
the secondary database would be unable to serve additional traffic while being
replicated. No record-replay system fits all use cases.

We explore four record-replay systems with different semantics enabling different use cases.
We first present \scribe, an OS-level deterministic record-replay mechanism
that support multi-process applications on multi-core systems. One of the main
challenge is to record the interaction of threads running on different CPU cores
in an efficient manner.
\scribe introduces two new lightweight OS mechanisms, rendezvous point and sync
points, to efficiently record nondeterministic interactions such as related
system calls, signals, and shared memory accesses. \scribe allows the capture
and replication of hard to find bugs to facilitate debugging and serves as a
solid foundation for our two following systems.

We then present \racepro, a process race detection system to improve
software correctness. Process races occur when multiple processes access shared
operating system resources, such as files, without proper synchronization.
Detecting process races is difficult due to the elusive nature of these bugs,
and the heterogeneity of frameworks involved in such bugs.
\racepro is the first tool to detect such process races.
\racepro records application executions in deployed systems, allowing offline
race detection by analyzing the previously recorded log. \racepro then replays
the application execution and forces the manifestation of detected races to check their
effect on the application. Upon failure, \racepro reports potentially harmful
races to developers.

Third, we present \dora, a mutable record-replay system which allows a recorded
execution of an application to be replayed with a modified version of the
application. Mutable record-replay provides a number of benefits for
reproducing, diagnosing, and fixing software bugs. Given a recording and a
modified application, finding a mutable replay is challenging, and undecidable
in the general case. Despite the difficulty of the problem, we show a very
simple but effective algorithm to search for suitable replays.

Lastly, we present \synapse, a heterogeneous database replication system
designed for Web applications. Web applications are increasingly built using a
service-oriented architecture that integrates services using a variety of
databases. Often, the same data, needed by multiple services, must be
replicated across different databases and kept in sync. Unfortunately, the data
replication engines of these databases are vendor specific and not compatible with
each other. To solve this challenge, \synapse operates at the application level
to access a unified data representation through object relational mappers.
Additionally, \synapse leverages application semantics to replicate data
with good consistency semantics using mechanisms similar to \scribe.
