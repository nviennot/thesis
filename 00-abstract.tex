\begin{abstract}

As applications running on traditional operating systems
grow in complexity, software bugs have become increasingly
common and more difficult to reproduce, diagnose, and fix.  Aggressive release
schedules exacerbate the problem, resulting in frail software that requires
patches to fix problems that occur in the field. Resolving a bug typically
starts with reproducing it in a controlled environment. Because the common
approach of conveying a bug report is often inadequate for tricky,
nondeterministic bugs, record-replay has been developed to capture application
bugs as they occur and deterministically replay the bug at a later time,
removing the burden of repeated testing to reproduce the bug.

First, we present Scribe, an OS-level record-replay mechanism.
Scribe introduces new lightweight OS mechanisms, rendezvous and sync points, to
efficiently record nondeterministic interactions such as related system calls,
signals, and shared memory accesses. Rendezvous points make a partial ordering
of execution based on system call dependencies sufficient for replay, avoiding
the recording overhead of maintaining an exact execution ordering.  Sync points
convert asynchronous interactions that can occur at arbitrary times into
synchronous events that are much easier to record and replay.

Second, we present Racepro, a process race detection system to improve
software correctness. Process races occur when multiple processes access shared
operating system resources, such as files, without proper synchronization.
Racepro leverages a low overhead record-replay mechanism to detect harmful races
as it deployed systems as opposed to synthetic environments.

Third, we present Dora, a mutable record-replay system which allows a recorded
execution of an application to be replayed with a modified version of the
application. This feature, not available in previous record-replay systems,
enables powerful new functionality. In particular, Dora can help reproduce,
diagnose, and fix software bugs by replaying a version of a recorded application
that is recompiled with debugging information, reconfigured to produce verbose
log output, modified to include additional print statements, or patched to fix a
bug.

Lastly, we present Synapse, an heterogeneous database record-replay system
specifically designed for web applications. These web applications behave very
differently compared to traditional single-host applications as they are
distributed and have their state contained in databases.
Synapse lets independent services cleanly share data with each other in an
isolated and scalable way. The services run on top of their own databases,
whose layouts, and engines can be completely different, and incorporate
read-only views of each others' shared data.  Synapse synchronizes these views
in real-time using a new scalable, consistent replication mechanism that
leverages the high-level data models in popular MVC-based Web applications to
replicate data across heterogeneous databases.

These four record-replay systems share many similarities and have many useful
applications, including auditing, debugging, testing, and bug detection.

\end{abstract}
