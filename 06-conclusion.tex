\chapter{Conclusion and Future Work}
\label{ch:conclusion}

\section{Conclusion}

We presented four different record-replay systems to achieve different goals.
We introduced \scribe, an OS-level deterministic execution record-replay system
to capture and reproduce hard to find bugs; \racepro, a process-race detection
system to improve software correctness in deployed systems; \dora, a mutable
replay system to reproduce nondeterministic bugs with retroactive debugging;
and \synapse, a replication system for distributed
applications to facilitate the use of heterogeneous databases in
service-oriented architectures.
Throughout each system we presented, we gradually moved away from faithful
replay, opening the range of use cases.

First, we presented \scribe{}, the first operating system mechanism to provide
transparent, deterministic execution record and replay of multi-threaded and
multi-process applications on commodity multiprocessors and operating systems.
\scribe{} records and replays multiple processes by accounting for
nondeterministic interactions among processes and their execution environment.
\scribe{} introduces {\em rendezvous points} to ensure correct partial ordering
of execution based on system call dependencies, and {\em sync points} to convert
asynchronous interactions that can occur at arbitrary times into synchronous
events that are much easier to record and replay.  \scribe{} can transition an
application  to running live at any time, and use checkpoints to record and
replay from any point in time.
We have implemented \scribe{} without changing, relinking, or
recompiling applications, libraries, or operating system kernels, and
without any specialized hardware support. It works on commodity Linux
operating systems, and commodity multi-core and multiprocessor
hardware.  Our evaluation shows for the first time that an operating
system mechanism can correctly and transparently record and replay
multi-process and multi-threaded applications on multiprocessors.  The
evaluation also provides strong empirical evidence that 
real server and desktop applications perform frequent
operating system activities which can serve as sync points.
\scribe{} recording overhead is modest for server applications including Apache
and MySQL, and for desktop applications including Firefox, Acrobat, OpenOffice,
parallel kernel compilation, and movie playback.

Second, we have presented the first study of real process races, and the first
system, \racepro, for effectively detecting process races beyond \toctou
and signal races.   Our study has shown that process races are
numerous, elusive, and a real threat.  To address this problem, \racepro
automatically detects process races, checking deployed
systems in vivo by recording live executions and then checking
them later.  It thus increases checking coverage beyond the
configurations or executions covered by software vendors or beta
testing sites.  First, \racepro records executions of multiple processes
while tracking accesses to shared kernel resources via system
calls. Second, it detects process races by modeling recorded system
calls as load and store micro-operations to shared resources and
leveraging existing memory race detection algorithms.  Third, for each
detected race, it modifies the original recorded execution to
reproduce the race by changing the order of system calls involved in
the races.  It replays the modified recording up to the race, allows
it to resume live execution, and checks for failures to determine if
the race is harmful.
\racepro heavily relies on record-replay mechanisms to perform its tasks.
Its engine extends \scribe's to allow specific modifications at the end of
a recorded log file.
We have implemented \racepro, shown that it has
low recording overhead so that it can be used with minimal impact on
deployed systems, and used it with real applications to effectively
detect \nracepro process races, including several previously unknown
bugs in shells, databases, and makefiles.
While \racepro's engine supports minimal modifications at the end
of a recorded log file, it must go live once these modifications are
executed during the replay due to divergence issues. We explored
how to alleviate these issues with \dora.

Third, we introduced the concept of mutable replay, and the first transparent
mutable record-replay system with \dora, allowing a recorded execution of an
application to be replayed with a modified version of the application. This
feature, not available in previous record-replay systems, enables powerful new
functionality. In particular, \dora can help reproduce, diagnose, and fix
software bugs by replaying a version of a recorded application that is
recompiled with debugging information, reconfigured to produce verbose log
output, modified to include additional print statements, or patched to fix a
bug.  This is made possible by the use of lightweight operating system
mechanisms to record and replay without imposing unnecessary timing and ordering
constraints.  \dora introduces an explorer that directs the replay mechanism to
identify a mutable replay of the modified application that minimizes differences
with the original unmodified application execution.
\dora's feature set is a superset of \scribe's. When the replayed application
has not been modified, \dora behaves similarly to \scribe. Thus, mutable replay
supports at least all use cases deterministic replay offers.
To implement mutable replay, \dora extends \racepro's record-replay engine
that paved the way to enable modifications in a recorded execution.
Our experimental results on a Linux prototype demonstrate that mutable
replay is feasible across a wide range of real-world applications and
application changes which can reach thousands of lines of code, even
without support for major changes to core application semantics.  We
show that mutable replay is useful for enabling common debugging
techniques not possible with previous record-replay systems.  We
also show that mutable replay enables validation of security patches
against both exploits and production workloads. This is all
accomplished without requiring source code modifications and with low
recording overhead, enabling usage on production systems.  These
results demonstrate that mutable replay has the potential to enable
new techniques for debugging and patch testing and validation, which
can lead to substantial improvements in software reliability and developer
productivity.
Until this point, we only explored OS-level record-replay mechanisms.
These mechanisms have limited support for distributed applications such as web
applications.

Lastly, we presented \synapse, an heterogeneous database replication system
specifically designed for web applications in a service-oriented architecture.
These applications run on top of their own databases,
whose layouts, and engines can be completely different, and incorporate
read-only views of each others' shared data. \synapse synchronizes these views
in real-time using a new scalable, consistent replication mechanism that
leverages the high-level data models in popular MVC-based Web applications to
replicate data across heterogeneous databases.
This replication is performed with record-replay techniques similar to
the ones introduced with our OS-level record-replay engines.
Execution is partially ordered by DB object access sequence numbers,
in a similar way our OS-level record-replay engines order OS resources accesses.
\synapse leverages models and ORMs to perform data integration
at data object level, which provides a level of compatibility between both SQL
and NoSQL DBs. It leverages controllers to support application-specific
consistency semantics without sacrificing scalability.  We have implemented
\synapse for Ruby-on-Rails, shown that it provides good performance and
scalability, and deployed it in production for a company.

We have shown four systems implementing transparent application execution
record-replay to achieve various use cases. These four systems are presented
in an order where replay faithfulness decreases.
With \scribe, we explored an implementation of deterministic record-replay where
the replayed execution appears to be identical to the original one from an application
perspective, but in reality, differs due to differences in scheduling to improve performance.
With \racepro, we showed how to apply minor changes at the end of a
recorded execution and replay such changes before going live.
With \dora, we introduced mutable replay which allow a modified version
of the same application to be replayed.
With \synapse, we achieved heterogeneous database replication with
record-replay mechanisms implemented in web frameworks.
Our findings showcase numerous
use cases of record-replay, ranging from (1) transparent fault tolerance
by recording a primary and replaying on replicas, (2) dynamic application
analysis by performing costly instrumentation on replicas that replay
application behavior recorded on production systems, such as process-race
detection, (3) debugging applications by capturing hard-to-find bugs and
reproducing them with recompiled and reconfigured version of applications to
greatly facilitate debugging, and (4) heterogeneous database replication.
We gradually relaxed the faithfulness of the replay throughout our systems.

Controlling the emergence of new executions during replay enables powerful new
use cases.

\section{Future Work}

The work developed in this dissertation raises the possibility for a number of
improvements and challenging questions to consider for future research
directions.
While \scribe shows excellent recording overhead in many applications, it can
exhibit high overhead with applications where multiple threads write at a
very high rate to the same memory page, causing page ownership to ping-pong
between threads degrading performance significantly.
Page ownership ping-ponging may also occur due to false data sharing among
threads. For example, if two private objects of two different threads are placed
on the same page, \scribe would serialize accesses to these objects even though
these objects are private to each thread.
To alleviate this overhead issue, applications could cooperate with \scribe: instead
of recording page access ordering transparently at the kernel level,
applications could inform the recording engine about which objects are accessed.
This cooperation can be done in two ways. With the first one,
\scribe still performs memory tracking as usual, except that the recorded
application is modified in a way to acquire and release pages ownership in
specific hot code paths. In addition, the application would have to store
unrelated objects on different pages to avoid false sharing effects.
This method is tolerant to application bugs as the kernel would still serialize
all memory accesses even when hints given by the applications are incorrect.
The second method is to offload the serialization of all object accesses to the application
Since the application should already have serialization primitives to protect
multiple writers accessing the same object, this mechanism is analogous to
\scribe's rendezvous points, but operating with the application objects as
opposed to kernel objects. This mechanism would thus have a very modest overhead,
despite high contention, and have no false sharing issues.
However, if developers misuse these user-space
rendezvous points, \scribe would no longer be able to guarantee deterministic
replay, as opposed to the first method. This second method is thus more suitable
to implement in runtime environments, such as the Java Virtual Machine (JVM).  In
these cases, assuming the runtime environment is correctly implemented, \scribe
would be able to guarantee deterministic replay with a much lower overhead.

As an increasing number of applications are using higher level languages running
atop of virtual machines, such as the JVM, the Ruby runtime, Python's interpreter,
the V8 engine for JavaScript to name a few, it becomes natural to offload \scribe's
recording engine of certain tasks. As discussed earlier, using user-space
rendezvous points is beneficial for performance, but would also allow
better application semantics extraction for use cases such as \racepro or \dora.
Other tasks can be offloaded, such as the interaction tracking of the runtime
environment with the kernel. For example, certain user-space APIs could cooperate
with \scribe to disable kernel-level recording of certain system calls, while performing
its own recording. For example, when performing an HTTP request, only the request
headers and body could be recorded, as opposed to recording the low-level
interaction with the underlying sockets. In this case, achieving deterministic
replay is still guaranteed while the recorded log no longer needs to contain the
information of doing DNS lookups, whether a keep-alive connection was used, if
the connection was redirected through 302 status codes, or if the stream was
encrypted through SSL.
This hybrid approach can be done in many ways as user-space applications often
have deep call stacks spawning multiple libraries. For example, in Ruby, a call
to \code{Net::HTTP.get(URI("www.google.com"))} invokes a few libraries written
in Ruby, invoking the Ruby runtime, then libresolv, libnss, zlib, openssl, which
in turn rely on libc to finally invoke around 50 system calls.
Using a hybrid approach is versatile as one could instrument a specific set of APIs
while retaining the original \scribe behavior for other unrelated APIs. This way,
deterministic recording guarantees can still be preserved, even when a high-level
application uses a non-instrumented C extension.
Tools that leverage record-replay mechanisms such as \dora and \racepro can
benefit hugely from these higher-level of recording abstractions as the
recorded log contains a more precise representation of application semantics.

\racepro benefits directly from this hybrid approach. Since \racepro's race
detection engine feeds directly from recorded object accesses via rendezvous points,
using user-space rendezvous points would allow \racepro to detect not only process
races, but also application races. It would be able to do so much faster
as the recorded log would be smaller and contain a lot fewer rendezvous points.

\dora is currently limited as it explores program execution
by removing and adding system calls. Leveraging application semantics allows
much better exploration. For example, consider a recorded application that
performs two HTTP GET requests that are sharing the same socket through HTTP
pipelining. When operating at the kernel-level, \dora is not be able to
find a good mutable replay as it would have to understand the underlying HTTP
protocol to discard the appropriate data from the received buffer on the
corresponding socket, which may be infeasible when using SSL connections.
On the other hand, when recording at the HTTP API level, \dora would be able to
just simply discard that one GET request from the recording to find the optimal
replay. By using a hybrid recording approach, \dora not only would be able to
find better mutable replays, but would do so much quicker as there are
less nodes to explore in the execution graph.

Additionally, both \racepro and \dora would benefit from a in-memory cloning
mechanism applied to an entire process container, with copy-on-write optimizations.
Indeed, both engines replay many executions that are identical from their start
up to a certain point. A lot of CPU is wasted replaying many times the same
execution up to that point. For example, when \racepro explores a race after
that point, it has to replay an entire execution up to that point, and then to
the race to be analyzed. Having checkpoints to replay from during the original
execution would greatly improve performance as the majority of the CPU is wasted
replaying the beginning of the execution. Similarly, \dora replays entire
executions, while it would be much more efficient to replay from the point of
divergence. Implementing an in-memory cloning mechanism entails a substantial
amount of work as many kernel objects would have to implement a copy-on-write
mechanism.

Earlier, we introduced \synapse which does record-replay in a
distributed environment.
While \synapse records all application state modifications by
interposing on the database layer, it does not record enough
information to deterministically replay an execution. For example, \synapse does
not record any of the incoming HTTP requests made to the application.  Together
with a hybrid record-replay implementation of \scribe, a useful distributed
deterministic record-replay system can be built.  While previous work has
explored the concept of distributed deterministic record-replay, it has failed
to provide useful semantics to build higher level tools. For example,
DDOS~\cite{ddos} records information at the packet level, and is not suitable
for high level frameworks used in web applications.  With a hybrid record-replay
system based on \scribe and \synapse, we could build a distributed deterministic
record-replay system for web applications suitable for implementing higher level
mechanisms such as race detection and mutable replay.

This hybrid record-replay system would complement the original premise of
\synapse which was to cleanly share data among services. \synapse currently
sends application state modifications from publishers to subscribers through a
message broker. The new hybrid system could publish additional messages
interleaved with the existing state modifications messages, providing necessary
information to do deterministic record-replay. For example, messages would include
received HTTP requests, or API call responses from third-party services such as
payment gateways. Recording data based nondeterminism is easy considering that
these data accesses are typically made through well-defined and simple APIs
(unlike a UNIX kernel API). However, recording time based nondeterminism is hard.
Thankfully, the underlying mechanism of the causal delivery mechanism of \synapse
is a mechanism similar to \scribe's rendezvous points allowing record-replay.
The hybrid record-replay system can record both data- and time-based sources
of nondeterminism in an efficient manner, send the recorded log to the message
broker alongside with regular \synapse messages, and let subscriber services
implement useful tools atop of record-replay.

Being able to subscribe to message streams from publishers that contain enough
information to perform deterministic record-replay allows, in addition to all the traditional \synapse
use cases, a set of new possibilities. A subscriber service can be deployed to
perform race detection in the background with a similar engine to \racepro.
Race detection can be very useful to save money. For example, \crowdtap suffered
from a bug in an open-source application they used for their e-commerce store
which allows users to redeem their points for Amazon gift cards.  The application
was decrementing the inventory item quantity by reading the quantity and writing
it back in a read-committed transaction (the default in most SQL databases) as
opposed to a serializable transaction. The race manifested in giving away
thousands of Amazon gift cards in excess. Having a subscriber service that
detects races in the background could be valuable to improve software
reliability.
Further, a subscriber akin to \dora can be deployed to implement mutable replay.
This can be useful to reproduce and debug elusive bugs. Adding logging after-the-fact
can be very useful. At \crowdtap, a bug resulted in throwing away acquired
gift cards due to a faulty retry loop in the gift card acquisition process.
The bug was discovered when the accounting team confronted their Amazon bills
with the numbers of given gift cards. Sadly, the developer implemented this
feature without logging any requests which made the recovery of gift cards
difficult. It would have been useful to replay the application with a code
modification allowing logging of gift cards.
To summarize, useful services can be implemented on top of deterministic
record-replay mechanisms for distributed web applications. We discussed race detection
and mutable replay, but other use cases could be envisioned, like performing
after-the-fact metric extraction, or performance analysis.
Further, these added-value services could be implemented by third-party
companies which would consume the message stream from the existing \synapse
message broker of the application.

In addition to backend record-replay, it would be useful to perform frontend
record-replay as an increasing number of web applications are deploying
substantial amount of logic in user devices. This avenue has already been
explored by Mugshot~\cite{mugshot}, a lightweight record-replay engine for
JavaScript that can run on unmodified browsers. By combining the recording
of the client application and the backend application, a developer could replay
a user session entirely from the frontend interactions down to the database,
including possible recorded races. This would provide the ability to reproduce
user behavior on the frontend, introspect interactions with other users through
backend services, and provide an accurate time machine for after-the-fact auditing
and code modifications.

In our experience, the success of mutable replay heavily relies
on the quality of the semantics extracted from the recorded execution.
The ability to distil core application state changes depends
on the abstraction level at which data is recorded. For example, recovering
meaningful application state changes by analyzing a hardware level recording
would be near impossible. On the other hand, if the application is written in a
way where state changes are explicitly described, usually in the form of
immutable domain logic events (e.g. \code{event(UserSignUp, params)}),
mutable replay becomes trivial to perform. An incarnation of this design pattern
is event sourcing~\cite{event-sourcing}. In the context of Web applications,
Event~Store~\cite{event-store} is an example of database with native support for
such design pattern. Redux~\cite{redux} is an example of application state
store for frontend applications. The latter is particularly interesting
as it already provides mutable replay features allowing hot code reloading and
time travel capabilities~\cite{redux-hot-reloading}.

To conclude, we only scratched the surface of what we can do with record-replay
mechanisms. Applications are increasingly built with high level abstractions, on
heterogeneous machines, with complex distributed semantics. Our current set of
record-replay tools are incapable of providing value for these applications. We
discussed feasible research avenues to build tools suitable for these
distributed applications.
