\chapter{Transparent Mutable Replay for Multicore Debugging and Patch Validation}
\label{ch:dora}

\section{Introduction}

As applications grow in complexity, software bugs have become
increasingly common and more difficult to reproduce, diagnose, and fix.
Aggressive release schedules
exacerbate the problem, resulting in frail software that requires patches to
fix problems that occur in the field. Resolving a bug typically starts with
reproducing it in a controlled environment. Because the common approach of conveying a bug
report is often inadequate for tricky, nondeterministic bugs, record-replay has been developed to
capture application bugs as they occur and
deterministically replay the bug at a later time, removing the burden of
repeated testing to reproduce the bug.

Despite an abundance of research on using record-replay systems for 
debugging~\cite{idna:vee06,instant-replay,r2:osdi,odr:sosp09,pinsel:pldi07,pres:sosp09,jockey,srinivasan:flashback,subhraveti:sigmetrics11}, these works have
focused on bug reproducibility and have had limited or no support for diagnosing
and fixing bugs.
Debugging almost always requires modifying the program, whether by
adding print statements, testing if a change fixes the problem, or
some other method.
However, most previous record-replay systems do not allow the recorded execution
to be replayed with any modifications to the application.
A handful of systems do allow some new code to be run in the middle of a replay,
but they do not support changes to the application
state~\cite{intrusions:sosp05,decouple:usenix08},
which limits the utility of these systems for debugging and validating changes.

To address this problem, we introduce {\dora}, a mutable
record-replay system which allows a recorded execution of an
application to be replayed with a modified version
of the application. Mutable record-replay provides a
number of benefits for reproducing, diagnosing, and fixing software
bugs. For instance, mutable record-replay can replay a version
of the recorded application that is recompiled with debugging information,
reconfigured to produce verbose log output, or modified to include additional
code instrumentation such as print statements.

Mutable record-replay can also replay a recorded
application execution of a production workload using
a patched version of the application. This is
useful for both application developers and system administrators.
An application developer can use a recording of a bug when developing
a fix. Replaying the recording on a modified application speeds up
debugging and provides a novel way of validating
bug fixes for nondeterminstic bugs, which can otherwise be time
consuming and difficult.
For example, a developer who writes a patch
can test it by taking a recorded execution of the exploit on the
original application and replaying it using the patched application to
quickly verify that the patch closes the vulnerability instead of
painstakingly regenerating the exploit for each attempted fix
of the problem.

System administrators often worry that applying patches
will break their applications.  Mutable replay allows administrators
to independently test patches on production workloads.  An
administrator can record the unpatched application in production,
apply the patch to an offline version of the application, and then replay
the recorded execution using the patched application.  If the
replay succeeds, the administrator will be more confident that
the changes will not introduce regressions.

Mutable replay complements traditional quality assurance
testing.  Quality assurance provides broad coverage but fails to
handle many corner cases, which is why bugs arise in production in the
first place.  In contrast, mutable replay isolates
actual bugs that occur in production. These bugs can be timing and
configuration dependent, so some surface very rarely. This coverage is often
not possible with traditional testing
due to nondeterministic program behavior.  Furthermore, mutable replay
provides fast turnaround time, enabling a bug to be replayed quickly
and directly tested against application changes that attempt to fix
the problem.
This not only speeds up debugging, but also provides a
way to validate bug fixes for nondeterminstic bugs, which can
otherwise be much more time consuming and difficult.

{\dora} consists of three components: (1) a recorder that records
application execution to a log, (2) a replayer that can replay a
modified version of the application using the log, and (3) an explorer
that uses the replayer to find the execution of the modified program
that best corresponds to the log file. The recorder and replayer build
upon our previous deterministic record-replay engine,
Scribe~\cite{scribe:sigmetrics10}.

The recorder operates primarily at the interface between applications and the
operating system (OS) to transparently record an application's nondeterministic
interactions. It avoids imposing unnecessary timing and ordering constraints
that would hinder mutable replay with a modified application.  To aid mutable
replay, the recorder also logs deterministic interactions to detect and
resolve any differences between the recorded application execution and the
replay of a modified version of the application.

The replayer replays a previously recorded execution using a modified version
of the application, matching events from the original
log with the actions of the modified program.
If the application used for replay is the same as the one recorded, the replayer
provides deterministic replay of the unmodified application.  However,
if the replayed application's behavior diverges from the original's,
the replayer gathers information for the explorer about the new code
path the program was trying to execute and waits for instructions on
how to proceed.  Because it operates at the OS level like the
recorder, the replayer has access to sufficient OS semantics to
understand why a replay diverges from the original execution and can
leverage these semantics to help the explorer.

The explorer evaluates several possible execution paths to find a successful
mutable replay.  It performs a best-first search for an execution of
the modified program that is as close to the original execution as
possible according to some cost function $d$. It begins by replaying a
recorded execution on a modified program. When the replay diverges
from the original execution, the explorer tries to determine
why. For example, suppose the modified program made an unexpected
\code{printf()} call. This could be a new call to produce debugging
information, or it could simply occur earlier than expected because
code was deleted. The explorer chooses the most promising possibility
and communicates its decision to the replayer. This process repeats
until a successful execution is found.

{\dora} is designed to handle an wide range of real-world programs,
including multi-threaded applications. It can support a broad range
of useful application changes, but cannot support arbitrary
changes; major changes to the process layout or shared memory
layout are not supported. Despite this limitation,
{\dora} is useful in a wide range of real-world use cases for
testing, debugging, and validating application changes.  In fact, we
even found a previously unknown bug in Apache using {\dora}~\cite{apache-bug-53131}.
{\dora}'s usefulness in practice makes sense given
that bug fixes tend to be relatively small and rarely change core
application semantics~\cite{delta,mreplay-feas}.

We have implemented a {\dora} Linux prototype that runs on
commodity multicore hardware without changing, relinking, or
recompiling applications or libraries.  Our experimental results with
over thirty different application changes show that {\dora} can (1)
record unmodified real-world multi-threaded applications with less
than 10\% overhead on multicore hardware, (2) replay applications that
have been reconfigured to produce verbose debugging output or modified
with added debugging instrumentation, (3) replay real exploits on
patched applications to verify that the patches close these
vulnerabilities, and (4) replay benchmark workloads to validate
application patches and version upgrades despite changes in thousands of lines
of code.

	We present the design, implementation, and evaluation of the
	{\dora} mutable replay system.
	Section~\ref{dora:sec:definition} provides a definition of mutable replay.
	Section~\ref{dora:sec:recorder} describes the {\dora} recorder.
	Section~\ref{dora:sec:replayer} describes the {\dora} replayer.
	Section~\ref{dora:sec:explorer} describes the {\dora} explorer and presents an
	example illustrating the use of the system.
	Section~\ref{dora:sec:properties} presents some key properties that
        can be guaranteed regarding
	{\dora}'s mutable replay behavior.
	Section~\ref{dora:sec:limitations} discusses limitations of the current system.
	Section~\ref{dora:sec:results} presents experimental results.
	Section~\ref{dora:sec:related} discusses related work.
	Finally, we present some concluding remarks and directions for future work.

\section{Mutable Replay Concept}
\label{dora:sec:definition}

Since mutable replay is a previously undefined concept, we begin by
presenting a definition.  Let $e$ be the recorded execution of some
program $P$. Let $E'$ be the set of possible executions of $P'$, a
modified version of $P$. A mutable replay of $e$ on $P'$ will then be
an execution $e'$ in $E'$ such that the differences between $e$ and
$e'$ are a result of the differences between $P$ and $P'$.

The difference between two programs includes not only differences in
their executables, but can also include changes in input and
environment, such as environment variables, the file system,
and host-related information.
Note that there is
not always a clear mapping from input in the original program to input
in the modified program. For example, the original program could read more bytes
from \code{stdin} than the modified program.

Since some executions in $E'$ are intuitively preferable to others, we introduce the
concept of a \emph{$d$-optimal mutable replay}, an execution in $E'$
that is optimal according to a cost function $d$.  The cost function
measures the difference between the original execution
and the mutable replay.  The value returned by $d$ reflects the minimal cost
of transforming the execution $e_1$ into a candidate execution
$e_2$. A lower score is better, scores can be negative, and the score
must be the lowest when $e_1$ is identical to $e_2$. A $d$-optimal
mutable replay $e_{d}$ of an execution $e$ on $P'$ satisfies $d(e, e_{d}) = min_{e' \in E'}
d(e, e')$. There is always at least one $d$-optimal mutable replay for
a given execution $e$ and a program $P'$.

Finding the $d$-optimal mutable replay is undecidable in the general
case. To show this, we first observe that finding the $d$-optimal mutable
replay requires running $P'$ because predicting the executions of a
program is undecidable. Suppose $P'$ has an added infinite loop at its
beginning. Then, when running $P'$, the replayer will loop infinitely
since detecting an infinite loop is undecidable. Thus, finding the $d$-optimal
replay is undecidable.

Even in the subset of cases in which finding a $d$-optimal mutable
replay is decidable, it is still NP-hard with respect to the number of events in
the log. Consider a program $P'$
which only adds a \code{read()} system call of \code{n} bytes. There are
$O(2^{n})$ possible results of this call. In addition, arbitrary signals could be
delivered between any two instructions. If the program is threaded,
they are many possible thread interleavings. Since
differences in signal delivery and thread interleaving could
theoretically cause radically different behavior, and since determining the
future execution of a program is undecidable, a mutable replayer must
consider every possibility, which is infeasible. Thus, no mutable replay system
can efficiently find a $d$-optimal mutable replay in all cases.

Fortunately, however, many useful changes to programs are modest in size and scope.
In particular, bug fixes tend to be relatively small and rarely change
core application semantics~\cite{delta,mreplay-feas}.  The same is
typically true of code instrumentation added to a program for
debugging. Based on this observation, we designed {\dora} with a
$d$ function that has useful properties for testing and debugging.  In
this context, Section~\ref{dora:sec:results} shows that {\dora} is able
to find $d$-optimal mutable replays in practice using real-world
applications. Furthermore, Section~\ref{dora:sec:properties} presents
guarantees that can be made about the optimality of {\dora}'s
approximation algorithm.

\begin{figure}
\begin{center}
\begin{footnotesize}
\begin{verbatim}
int main() {
  printf("d\n", time(NULL));
  return 0;
}
\end{verbatim}
\end{footnotesize}
\end{center}
\caption{Original program}
\label{dora:fig:orig-code}
\end{figure}

\begin{figure}
\begin{center}
\begin{footnotesize}
\begin{verbatim}
int main() {
  FILE *out = fopen("output", "w");
  fprintf(out, "d\n", time(NULL));
  return 0;
}
\end{verbatim}
\end{footnotesize}
\end{center}
\caption{Modified program}
\label{dora:fig:mod-code}
\end{figure}

A simple example may help further clarify the concept of
mutable replay.  Figure~\ref{dora:fig:orig-code} shows a program that
prints the current time in seconds to {\tt stdout}. Figure~\ref{dora:fig:mod-code}
shows a modified version of the original program that instead writes the output to a file.
Intuitively, we want the {\tt gettimeofday()} call in the replay of the
modified program to return the same time returned in the recorded
execution of the original program.
We will show that {\dora} does this, producing a $d$-optimal replay
with the cost function described in Section~\ref{dora:sec:explorer}. 

\section{Recorder}
\label{dora:sec:recorder}

{\dora}'s recorder builds upon our Scribe work on lightweight OS-level
deterministic replay on multiprocessors~\cite{scribe:sigmetrics10}.
The Scribe engine provides four key benefits for mutable replay. First, operating
at the OS level avoids tracking low-level hardware nondeterminism that
is unnecessary for application replay and would significantly
complicate mutable replay.  Second, {\dora} records the execution
of system calls in a manner that enables system calls and their effects
on the kernel to be fully executed during replay.  As discussed in
Section~\ref{dora:sec:replayer}, this is essential for mutable replay
because there are times when {\dora} must transition processes
from controlled replay to live execution to enable mutable replay.
Third, {\dora}'s recorder can record the execution of multiple
processes and threads with low overhead on production systems. Finally,
{\dora}'s recording is transparent to applications.  It does not
require changing, relinking, or recompiling applications or libraries,
and it supports programs written in any programming language.

The recorder operates on a group of tasks (threads and processes), which we
refer to as a {\em session}. {\dora} records interactions between the
session and its external environment, such as incoming network packets and
nondeterministic interactions between tasks, in a manner which accommodates
application changes during replay. {\dora} also records deterministic
information to help detect changes in an application's execution path during
replay.

The recorder saves the recorded execution to a log file.
Figure~\ref{dora:fig:example-orig-log} shows the
tail of the log file generated by running the simple program in
Figure~\ref{dora:fig:orig-code}. We excluded 83 events related to program initialization, including {\tt
execve()} and C library bootstrapping events. The last two initialization events
are shown. The {\tt rdtsc} event corresponds to seeding a random generator from the C library
by reading the time stamp counter of the CPU. The log also includes
several system calls and information about their arguments.

\begin{figure}
\begin{center}
\begin{footnotesize}
\begin{verbatim}
  // 83 initialization events
  --- cut ---
  munmap(0xb76e1000, 968e) = 0
  rdtsc = 000057ed322904cf
  time(NULL) = 0x4f9bd2e7
  fstat(1, 0xbff8d684) = 0
  mmap(0, 1000, 3, 34, -1, 0) = 0xb76ea000
  write(1, 0xb76ea000, 11) = 11
  exit(0) = 0
\end{verbatim}
\end{footnotesize}
\end{center}
\caption{Recorded log file of original execution}
\label{dora:fig:example-orig-log}
\end{figure}

\subsection{Nondeterministic Interactions}
To replay an execution deterministically, the timing and ordering of
nondeterministic interactions between tasks must be recorded.  To
enable mutable replay, {\dora} must also be resilient to changes in
the application behavior which add or remove nondeterministic system calls. We
highlight how key interactions involving system calls, signals, and shared
memory are handled.

\paragraph{System calls.}
The ordering of system calls that access shared resources causes nondeterminism
if at least one call modifies the resource.
For instance, a \code{write()} and a \code{read()} on the same
pipe are \emph{related system calls} since they access a shared resource and
their relative ordering matters.  Preserving the order of related
system calls during replay ensures that recorded racy behavior is maintained in
the replay.
To provide this deterministic behavior, {\dora} operates at the OS level to
capture concurrency at the same level of granularity as the OS.
{\dora} leverages
kernel code that already serializes access to
shared kernel objects to record and enforce a partial ordering of all related
system calls.
This differs from previous approaches~\cite{r2:osdi,srinivasan:flashback} 
which impose a total ordering of system calls.  Such approaches do
not scale well for multicore and do not work for mutable
replay. Application changes are likely to change the number and order
of system calls, making a total ordering of system calls too
restrictive for replaying an execution using a modified program.  

Shared objects tracked by {\dora} include inodes, files,
file-tables, memory maps, process credentials, process states, and
system-wide properties such as the hostname and mount points.
To keep track of access ordering, every resource is assigned a
globally unique identifier and a serial number indicating the order of access.
For each resource a system call accesses, {\dora} increments its serial
number and records its identifier and serial number. This allow
the replayer to deterministically reproduce the access sequence.

\paragraph{Signals.}
The delivery time of signals is another source of nondeterminism.
Signals are normally delivered in two steps in the kernel.  First, the sender process sends a
signal to the target process, which is marked as having a signal
pending.  Second, the target process detects and handles the pending signal when
it returns to user space from kernel space.  If the target process
is executing in user space on another core, an inter-processor interrupt will force it
into kernel space, where it will detect the pending signal.
Replaying this behavior can be challenging because it requires
interrupting the target process at the exact same instruction as
during its original execution.  Previous
approaches~\cite{bressoud-tft,bressoud,revirt,smp-revirt:vee08}
have relied on hardware providing a cycle accurate instruction
counter~\cite{replica-nondet:ftcs96}, but this does not work for CPUs
without such counters and does not work for mutable replay.
Application changes are extremely likely to change the number and
ordering of instructions, rendering the counter values useless.

To address this problem, {\dora} defers signal delivery to
locations called
\emph{sync points}.  Sync points are well-defined locations in an
execution which deterministically cause the process
to enter kernel space, such as system calls, traps due to division by
zero, and page faults due to shared memory accesses, as discussed
below.  By
only delivering signals at sync points,
{\dora} effectively converts asynchronous events into
synchronous ones and does not require special hardware or application
modifications. This also makes it much easier to replay using an application with
modified signals.
Note that this behavior complies with the POSIX
semantics of signals.

Delaying signal delivery to sync points may introduce some latency in
the application. However,
our previous work on Scribe~\cite{scribe:sigmetrics10} shows that
sync points occur quite frequently in common desktop and server
applications. Our evaluation showed that deferred signals were delayed by
less than 100 $\mu$s on average with a maximum delay of less than 1
ms, so this latency is generally imperceptible.  Furthermore, the vast
majority of signals were delivered immediately
because the target process was
already in a sync point when the signal was sent. This is
not surprising given that processes often stay in a sync point for a
prolonged period of time. For example, a process blocking on I/O in a system
call is in a sync point.
While a process is in a sync point, signals are delivered
immediately and need not be deferred.

\paragraph{Shared memory.}
Shared memory accesses cause nondeterminism arising from the order in
which threads and processes read and write to the same memory locations.
Memory is shared either implicitly, as with threads that share an
entire address space, or explicitly, as with processes that share a common
memory mapping.  Since processes typically access multiple locations
on a page during a given time interval because of spatial and temporal
locality, {\dora} tracks memory accesses at a page granularity
by managing page ownership with the help of the hardware page protection mechanism.
Each shared memory page is assigned an owner process or thread for
some time interval. The owner can exclusively modify that page during
that interval and treat it like private memory. Thus, {\dora} does not need
to track every memory access during such
periods. Transitioning page ownership from one process or thread to
another is done using a concurrent read, exclusive write (CREW)
protocol on memory pages with some optimizations~\cite{scribe:sigmetrics10}.

To ensure that ownership transitions occur at precisely the same
location in the execution during both record and replay, previous
approaches~\cite{bressoud-tft,bressoud,revirt,smp-revirt:vee08}
have relied on cycle-accurate hardware instruction counters.  An ordering
this precise is not resilient to application changes, so it does not work for
mutable replay. To address this problem, {\dora} defers ownership
transitions until the owner reaches a sync point.  As with system
calls accessing shared resources, a memory event is logged for the
thread that accesses shared memory. The event includes a page
identifier and a per page serial number indicating the order of access; this
process is similar to the one for system resources.

{\dora} implements this CREW protocol by creating a
\emph{shadow page table} for each thread, which is a private version
of the shared page table.   If a thread does not have ownership of a
page, its shadow page table entry (PTE) access privilege bits are
cleared.  When a thread tries to access a page owned by another
thread, it triggers a page fault, notifies the owner, and blocks
until access is granted.  Note that while the thread is blocked, it
is in a sync point, so it can immediately release pages to other
requesting threads, which prevents
deadlock~\cite{scribe:sigmetrics10}.  Once the page owner reaches a
sync point in its execution, it transfers the page ownership to the
thread requesting access, which then returns from its page fault
handler.
Recording performance can be negatively impacted for applications that
exhibit an enormous number of ownership transitions.  Nevertheless,
{\dora} performs well for a wide range of 
real-world applications as shown in Section~\ref{dora:sec:results}.

\subsection{Additional Information for Mutable Replay}

In traditional replay, all deterministic information need not be
recorded because it will be regenerated during replay, but in mutable
replay, the modified application may behave differently even in
deterministic sections of code.  {\dora}'s recorder stores
additional deterministic information in the log to help the replayer
detect differences between the original and replayed
executions as early as possible.  Since {\dora} provides deterministic replay for
unmodified applications, it does not need to record the additional
deterministic information in production, but instead records such
information afterwards by replaying the original
execution and recording additional deterministic information as
needed.

As part of this process, {\dora} records all system calls executed, not just those
involved in nondeterministic interactions.  {\dora} records the
system call number, arguments, and return value for each system
call. For arguments that are pointers, {\dora} follows the pointer
chain to record the actual memory contents, which are used during
replay to see if two system calls are equivalent.  By recording memory
contents instead of the pointer values, {\dora} is more tolerant of
memory layout changes caused by application modifications.

{\dora} also records the virtual addresses of shared memory accesses,
allowing the replayer to match shared memory access to detect divergence.
However, this mechanism means that changes to the memory layout of
writable shared memory affecting page
boundaries can cause {\dora} to incorrectly replay data races.
Fortunately, a recent study of common security patches indicates that
a vast majority of application
patches~\cite{mreplay-feas} do not make such changes.

Finally, when performing mutable replay, the modified application
may introduce system calls that are nondeterministic and environment
dependent.  {\dora} addresses this issue by recording two
additional types of information during the original recorded
execution.  First, {\dora} stores additional
information for mutable replay to ensure that nondeterministic actions
that occur during the modified application replay but not during the
original recorded execution are consistent with the recorded
execution.  For example, {\dora} periodically
records timing information to ensure that any new calls to time-related
functions are consistent both with each other and with any calls in the recorded execution.
Second, {\dora} records other information about the execution environment so
that new system calls in the modified application behave as they would have in
the environment in which the program was recorded.
For example, the recorder stores
host information in case the modified application requests it with a new
\code{uname()} or \code{gethostname()} call.
\section{Replayer}
\label{dora:sec:replayer}

{\dora}'s replayer replays the originally
recorded execution using either the original program or a modified
program. It requires that the execution is replayed on a machine which
supports all the recorded instructions. For example, a program that
uses SSE instructions when it is recorded cannot be replayed on a
machine without SSE instructions unless it is recompiled to use a different
ISA.  The replayer uses the recorded log file to generate a separate
log file per task.  Each task is replayed independently, but the
replayer enforces the recorded order of access to shared resources.  

A key aspect of the replayer is that it can
transition a task or a group of tasks from controlled replay to normal
execution. This feature is essential for mutable replay because a
modified program may have new code to execute that is not
part of the recorded execution.
{\dora} can run such code at any time because it
fully executes system calls and their effects on the kernel during
replay. Many other replay systems only emulate the effects they have
on userspace~\cite{jockey,r2:osdi}, but this would prevent normal
execution of the application from being enabled in the middle of
replay.

As a task executes kernel code, the replayer compares the execution
with what is expected in the log file.  When the execution
\emph{matches} expected events in the log, the 
replayer ensures it behaves as it did in the original execution.  
System calls match if they have the same system call numbers and
arguments. When an argument is a pointer to a buffer, {\dora}
compares the contents of the buffers instead of the pointer addresses.
Shared memory access events match if the access types and page
addresses are the same.  The replay ends successfully if all tasks
terminate after consuming every recorded event. A replay that uses an
unmodified application will always end successfully in this manner.

However, if a replaying task is about to execute a code path that does
not correspond to the expected event, the replay has \emph{diverged}
from the log.  The replayer conveys this to the explorer, which
determines how the replayer should resolve the divergence.
If the explorer determines that the replayer should continue
replaying the current log, the unexpected event can be treated in
number of different ways to try to resolve the divergence so that
later events will match events in the log.  For simplicity, {\dora}
treats a divergence as one of two possible types of \emph{mutations},
an \emph{addition} or a \emph{deletion}.  

\subsection{Additions}
An addition is an event added to the program.  For example, a program
could be modified by adding code that includes a new system call.  A
new event is most often a system call, but it can also be a new signal
or shared memory access.  When executing an event not in the log file,
{\dora} has three main responsibilities.  First, it must decide when
to execute the new event relative to events already in the log
file. System calls, signals, and shared memory events are often racy
with respect to other processes or threads, so there can be many
possible orderings.  Second, it must ensure that the semantics of the
event are consistent with the semantics of the recorded execution.
Finally, it is useful to be able to deterministically reproduce these
decisions in subsequent replays, as explained in
Section~\ref{dora:sec:explorer}. 

To handle these responsibilities, the replayer switches the process
that encountered the addition from controlled replay into 
\emph{direct mode}.  Direct mode switches the respective process to  
normal execution  and enables {\dora}'s recorder to record the execution.
This adds a new event to the log and orders it
with respect to other events in the log.
The
resulting log can then be later deterministically replayed with the
same modified program.  Once the process completes the additional
operation, it returns to regular replay mode; no other process has
left regular replay mode.  We discuss how this is done in further detail
for new system calls, signals, and shared memory accesses. 

\paragraph{System calls.}
When encountering a new system call, {\dora} executes the
call and records the execution as discussed in Section~\ref{dora:sec:recorder}.
This is possible because the replayer fully executes system calls and
their effects instead of just emulating them, ensuring that it is
possible to switch a process to normal execution at any time.
{\dora} further instruments various system calls to ensure that the
new system call behaves consistently with the recorded execution.
The specifics of this depend upon the semantics of the added system
call. We highlight system calls that deal with three important types of issues:
environmental or timing information, resource allocation, and sockets.

For system calls that request environmental information or timing
information, {\dora} ensures that the return values are made
consistent with the information in the original log.  This includes
{\tt gettimeofday()} and {\tt gethostname()}.  For example, if a new
{\tt gethostname()} call is executed, {\dora} already recorded such
environmental information and ensures that the name reported is 
the same as what was already recorded.

For system calls that request new resources, {\dora} ensures that 
assigned resources do not conflict with those used by the replayed
execution.  For example, if a new page in memory is allocated,
{\dora} ensures that its address will not conflict with those used
in the original program.  If the system call manipulates an existing
shared resource, the respective resource serial numbers are renumbered
to account for the new event. 

{\dora} simply executes new socket-related system calls during replay
except when dealing with data streams originating from outside the
session.  To deal with data streams, such as
external sockets, {\dora} registers fake backends to the
corresponding file descriptor.
Socket system calls related to those data streams will simply
manipulate the recorded network data.  For example, if a {\tt read()}
on a network socket is changed to a {\tt recvmsg()} on the same
socket, {\dora} provides the appropriate
data. If a new {\tt read()} from a socket tries to access
more data than recorded, $0$ is returned to indicate end of file. 

\paragraph{Signals.}
When encountering a new signal, such as from a modified application
with a new {\tt kill()} call, {\dora} needs to determine when to
deliver the signal to the target process.
The replayer does this much as the recorder does.
Like the recorder, the replayer
defers signal delivery until the target process encounters a sync point. This
ensures that signal delivery can be replayed deterministically during subsequent
replays.

\paragraph{Shared memory.}
When encountering a new shared memory access, {\dora} needs to
determine how to interleave the access with other accesses.
As in recording,
a page fault occurs
when a replayed
process tries to access a shared page that it does not own,
and the process must acquire
ownership of the page.  Once the process obtains ownership and
completes the memory access, it releases ownership to the previous owner at the
next sync point to ensure that the access order in the original execution is
respected.  The new memory events are added to the log file and the
original serial numbers are reordered as necessary to ensure that
subsequent replays based on this log deterministically perform the
memory access in the same way.

\subsection{Deletions}
A deletion corresponds to the removal of events from the original log
file and implies that the unexpected event matches a later event in the
log. The replayer deletes the intermediate events.  There can be
several possible matches for an event if the event occurs several
times later in the original log.  {\dora} identifies
possible matches and reports each possible match to the explorer. Because it is
expensive to process many events and it becomes increasingly unlikely
to find a match that will result in a successful mutable replay if an
extremely large number of events need to be deleted, {\dora}
imposes a cap on the number of events it can remove for a deletion.
The cap is 10,000 events in our implementation.  We present
further detail regarding deletions that involve system calls,
signals, and shared memory accesses.

\paragraph{System calls.}
Most system calls do not have any side
effects in the log file, so not executing a call itself is all that is
necessary to delete it. If the system call involves a shared resource, {\dora}
also renumbers the serial numbers for the resource so that its serial
number sequence does not contain any gaps. Deleting system
calls related to external sockets does not remove the incoming data, since it is
preserved as part of the stream of data associated with the resource.
Data that is not consumed from a deleted socket system call
will eventually be consumed by other remaining or new system calls.

When a deleted system call was originally associated with the
delivery of asynchronous events, the events must be relocated to other
sync points.  For example, consider a program that has a {\tt SIGALRM}
delivery scheduled on a {\tt getpid()} sync point.  If the new
execution no longer calls {\tt getpid()}, {\dora} must choose a new
sync point at which to deliver the {\tt SIGALRM} signal.  {\dora}
postpones the delivery of asynchronous events until the next sync
point the application encounters.
Choosing the next sync point is better than choosing the
previous one because releasing page ownership prematurely would
introduce spurious page faults in the application and could prevent
{\dora} from respecting the original page access order.  For
example, suppose a thread writes to a page and then releases ownership
of the page at the following sync point, which is on a system call. If
the system call is removed and {\dora} moved the ownership release
event to a previous sync point, it would occur before the access to
the page. This access would then trigger a page fault and generate a
new ownership acquisition event that may not respect the
original ordering.  Moving the release of ownership to the following
sync point avoids this issue.

\paragraph{Signals.}
Deleting a signal involves deleting its source, which is typically a
system call. System calls that deliver signals require additional
consideration because their effects create multiple events in the log
file.  For example, removing a \code{kill()} system call must also
remove the delivery of the corresponding signal.  During recording,
{\dora} associates delivered signals with their sources by using an
incrementing global token used across the entire session.
When a signal is about to be delivered, {\dora} waits for the
source to be triggered or deleted, which respectively delivers the
signal or omits the signal from being delivered. 

\paragraph{Shared Memory.}
When deleting a shared memory access, the corresponding ownership
acquisition event should not be executed.  However, the previous owner
still releases its page ownership so that its behavior is consistent
with the original recorded execution. Asynchronous events associated with a
deleted shared memory event are relocated to other sync points in the same
manner as they are for system calls.

\subsection{Going Live}
In rare cases, the entire log file is
consumed before the modified application terminates. This can occur when the
original application crashes, but the patched application avoids crashing. The
replayer allows the session to \emph{go live} and entirely transition from
controlled replay to live execution. This enables the user to
validate the correctness of the patch.
Since {\dora} faithfully replays kernel actions, the system is always in a state that allows
it to transition to live execution. Linux namespaces~\cite{namespaces:linux06}
create a consistent environment for processes before and after they go live.
Examples of this are demonstrated in Section~\ref{dora:sec:results}.
\section{Explorer}
\label{dora:sec:explorer}

The explorer uses the replayer to search for a $d$-optimal replay. When the
replay diverges, the explorer must determine how to proceed.
If this was the first divergence, the
explorer decides whether the replayer should consider the mutation as
an addition or a deletion. If there were previous divergences, the
explorer might also tell the replayer to reconsider a previously
detected divergence and explore a different path. In this case, the
explorer provides the replayer with a new log file.  Thus, the 
replayer needs no knowledge of the exploration algorithm.

The explorer treats the problem of finding the best mutable replay as
a search through a tree $T$ of possible candidate executions from a
start node $e$, the execution of the original program, to one of
many goal nodes, which represent executions in $E'$.
This problem is different from most other search problems because
(1) expanding a node can result in an infinite loop,
(2) there can be virtually infinite goal nodes, and
(3) the paths to the
goal nodes are not known beforehand because determining the possible
executions of a program in advance is undecidable.

Since an exact search is undecidable in the general case and NP-hard
even when it is decidable, {\dora} performs an inexact search using
a modified uniform-cost search.  For simplicity, {\dora} only
considers additions and deletions.  It does not consider, for example,
input fuzzing or trying all possible racy paths. The algorithm
performs the following steps: 

\begin{enumerate}
  \item{Initialize $T$ to contain the root node $e$.}
  \item{Pick an unexplored execution in the tree with the lowest cost according
	to the cost function $d$.}
  \item{Attempt to replay this execution on $P'$.}
  \begin{enumerate}
    \item{If this replay succeeds, this execution is selected and the exploration
	concludes.}
    \item{Otherwise, the replay diverges on an unexpected event, and new
	nodes are added to the graph. One node represents an addition
        and the others correspond to each possible deletion.
        Each node has an associated log
        file so that nondeterminism due to mutations is reproduced
        exactly across replays. Go to step 2.}
  \end{enumerate}
\end{enumerate}

A useful feature of the explorer is that the end result of a
replay up to a given node is recorded. Since this recording can be
deterministically replayed, a mutable replay is easily reproducible.
Furthermore, it is easy to compare two logs to see the differences
between two executions.  For example, a
developer can compare the log of the originally recorded execution with the log of
a mutable replay to understand how application modifications affected the
replay.

For simplicity, we have implemented the explorer algorithm by
replaying a new execution from the beginning of the log upon divergence.
In reality, there is no inherent reason for executions to be replayed from the
beginning since each child node's log only differs from its parent node's log
after the point of divergence. For example, a checkpoint could be taken just
before divergence occurs. Nodes created because of this divergence could
replay from the checkpoint instead of from the beginning of
execution~\cite{dejaview,zap07,zap02}.

While any function satisfying the properties specified in
Section~\ref{dora:sec:definition} can be used for $d$, we present a simple
function that has useful properties for debugging purposes.
Since matches are desirable and additions and deletions are undesirable,
each match has a cost of $-M$ and each addition or deletion has a cost of
$+1$, where $M > 1$. We use a value of $3$ for $M$ in our prototype,
but the process of selecting a mutable replay was relatively
insensitive to the specific value.
Using a negative cost for matches
means that the explorer is unlikely to backtrack after making many
contiguous matches.  This also means that the uniform-cost search
is not guaranteed to find the optimal replay even amongst the nodes it
considers (additions and deletions).  We made this decision because
the number of nodes it would need to consider to guarantee correctness
is exponential. Since the future execution of a program is 
unknown, even potential executions which seem very unpromising could
theoretically match many later events in the log file and obtain a very good
score. Thus, the search would effectively become a breadth-first search if $d$
could only return non-negative numbers.  Event logs may have billions of
events long, so this would not be feasible.

\paragraph{Example.}
To make this process more clear, we return to the example introduced in
Section~\ref{dora:sec:definition}, in which a program that prints the time to
{\tt stdout} is modified to write the time to a file.
Figure~\ref{dora:fig:diff} illustrates how the explorer replays the
modified program in Figure~\ref{dora:fig:mod-code} using the recorded execution
shown in Figure~\ref{dora:fig:example-orig-log} of the original program in 
Figure~\ref{dora:fig:orig-code}. The explorer starts by replaying the
original log file on the modified program. The complete original log file
contains 90 events, but Figure~\ref{dora:fig:diff} omits initialization
events, such as {\tt execve()} and bootstrapping code from the C library. 

\begin{figure}
\begin{center}
\begin{footnotesize}
\begin{verbatim}
  munmap(0xb76e1000, 968e) = 0
  rdtsc = 000057ed322904cf
+ brk(NULL) = 0x9870000
+ brk(0x9891000) = 0x9891000
+ open("output", 577, 438) = 500
  time(NULL) = 0x4f9bd2e7
+ fstat(500, 0xbff8d674) = 0
- fstat(1, 0xbff8d684) = 0
  mmap(0, 1000, 3, 34, -1, 0) = 0xb76ea000
+ write(500, 0xb76ea000, 11) = 11
- write(1, 0xb76ea000, 11) = 11
  exit(0) = 0
\end{verbatim}
\end{footnotesize}
\end{center}
\caption{Mutable replay of modified program}
\label{dora:fig:diff}
\end{figure}

The replayer matches the first 85 events successfully,
which are all system call events, resulting in a cost of -255.
At this point, the replayer encounters a {\tt brk()}
that does not match the {\tt time()} call in the original log
file and diverges. Upon divergence, {\dora} can treat {\tt brk()}
as an added system call or search the original log for a {\tt brk()} call and
delete intermediate events. Since no other call to {\tt brk()} occurs in the log
file, only the addition path is considered, resulting in a cost of -254 and an
additional node in the tree of candidate executions.

Following this algorithm, replayer adds two more system calls:
another {\tt brk()} and {\tt open()}, resulting in a cost of
-252.
A {\tt time()} call is executed and successfully
matches the expected call in the original log, ensuring that the time returned
in the replayed execution is the same as the time in the original log. The match
lowers the cost of the execution path to -255.

The changed program then executes an {\tt fstat(500, \ldots)} in the
replaying execution.  Although the system call number is the same as the {\tt
fstat(1, \ldots)} in the log, the file
descriptors passed as the first argument are different.  This is treated as a
mismatch.  No
subsequent matching {\tt fstat()} calls are in the log, so this is 
treated as an another addition, increasing the cost of the execution path to -254.

Next, the application diverges on the call to {\tt mmap()}
since it does not match {\tt fstat(1, \ldots)}. For the
first time in this example, the divergence can lead to a deletion or addition
since
there is a matching {\tt mmap()} in the original log. The explorer creates two
nodes, and explores the unexplored node with the lowest cost. In this case,
these two nodes are the only unexplored nodes. The
addition node costs -253 since there is a one point addition penalty. The deletion
node costs -256 because there is a one point deletion penalty for removing one
node and a three point bonus for matching {\tt mmap()}. Therefore, the deletion
node is selected. The addition and deletion of {\tt fstat()} is effectively a
\emph{replacement} of the system call.

The modified program then runs a {\tt write(500, \ldots)} in the
replaying execution which is different from the {\tt write(1, \ldots)}
in the log. Although the function calls are the same, the file
descriptors are different, so this is treated as a mismatch.  Since no
subsequent matching {\tt write()} calls are in the file, this must be
treated as an addition, increasing the cost of the execution path to -255.

Finally, the program calls {\tt exit()}, which diverges
from the {\tt write(1, \ldots)} in the original log. The
divergence can lead to an addition of {\tt exit\_group()}
or a deletion of {\tt write(1, \ldots)}, matching the
{\tt exit\_group()} in the original log.
The addition node costs -254, the deletion node costs -257, and the unexplored node
which added {\tt mmap()} costs -253. The deletion node is selected.

Since the end of the log file has been reached, the explorer has found a
successful replay with a cost of -257 and terminates. In this case, the explorer
has successfully found a $d$-optimal replay. Although the explorer cannot prove
this, the optimality of this replay is evident given the nature of the
application modification.

From this simple example, we can observe that small code
changes may significantly impact the behavior interactions of the application
with the kernel API.  The {\tt fopen()} library call internally calls {\tt
malloc()}, resulting in two new invocations to {\tt brk()}. Thus, even minor
changes to high-level source code can result in relatively large changes to the
low-level executable code.
\section{Properties}
\label{dora:sec:properties}

We can make several useful guarantees about the behavior of {\dora}
for certain classes of application changes.

\newtheorem{property}{Property}

\begin{property}
	{\dora} deterministically replays the original execution
        if the program is unmodified.
\end{property}
        In other words, {\dora} performs traditional
        deterministic record-replay when the program is unchanged.
        This property also implies that {\dora} provides
        $d$-optimal mutable replay for all $d$ for unmodified programs.

\begin{property}
	If all explored mutations are \emph{safe}, {\dora}
        deterministically replays all events in the original execution
        with the modified program. 
\end{property}
	For our $d$, a safe mutation is an addition that does not change
	any state which is read by the original execution. These additions may store
	state which is later read, but the original execution must not access this
	new data. This guarantee is quite useful for
	debugging because it implies that all behavior in the original program,
	including race conditions, will be preserved deterministically.

	For example, a \code{printf()} changes the internal state
	of the program by modifying an internal buffer, but returns the program to
	its original state, assuming proper newlines. Therefore, adding a {\tt printf()} to
	debug a race condition always preserves recorded races because
	it will not change the relative ordering of events in the
	log. {\dora} can also handle the creation of new files. Any new calls to
	\code{open()} will receive a file descriptor not used by the original execution,
	so {\dora} guarantees that the original behavior of the program will be
	unaffected.

	As another example, consider memory changes, for both shared and private
	regions of memory. First, reading the value of a variable in memory is safe.
	This is true even if the read is from shared memory and triggers a page fault
	leading to a temporary ownership transition. Additionally, the program can
	allocate and write to pages that are unused by the original
	execution. To avoid conflicts, {\dora} always assigns new memory
	allocations to a reserved area that is isolated from the original memory mappings.

\begin{property}
	{\dora} does not guarantee deterministic replay when given a modified application
	with arbitrary modifications.
\end{property}
	As explained in Section ~\ref{dora:sec:replayer}, {\dora} does not evaluate
	all possible interleaving of additions.
	For example, when a \code{printf()} is added in two different
	threads without locking, the order in which the calls are executed is
	variable. {\dora}
	picks the first possibility it encounters during replay and enforces this
	ordering for subsequent replays. However, this ordering is not enforced
	across separate invocations of the explorer.

\begin{property}
	{\dora} can deterministically replay a mutable replay of a
	modified application.
\end{property}
	{\dora}'s explorer outputs the replay it selects to a log file. Thus, {\dora} can
	deterministically replay the original execution of the explorer using the
	modified program. This enables exact reproduction of a previously found
	replay, allowing {\dora} to be used iteratively.
\section{Limitations}
\label{dora:sec:limitations}

{\dora} has several limitations as it has no knowledge of application
semantics.  As a result, it only supports replay across application changes
that do not alter core application or execution semantics.

For example, an exploit might add a new entry to a MySQL database. This entry
would be assigned a particular id by MySQL, and would affect the ids of all
later entries. A patch that removes the exploit would also remove the created
entry, resulting in a mismatch between the ids assigned during replay and the recorded assignments. {\dora} would not find a
$d$-optimal mutable replay because the core semantics of the execution have
changed.

Additionally, {\dora} currently does not effectively support major changes in the
layout of shared memory. If objects are relocated from a page to another, {\dora} cannot always
preserve the original access ordering since {\dora} manages
shared memory at the page level. {\dora} could be modified to track
objects instead of pages by instrumenting the application.
Even without this functionality, however, {\dora} is able to handle
some changes to MySQL, which heavily uses shared memory.

Finally, {\dora} currently does not support process/thread layout changes. For
example, if an application was originally recorded with 10 threads running, and
is now reconfigured to run with 5 threads as part of the application
modification, {\dora} does not provide a way to find a good mutable replay.
Similarly, {\dora} has difficulty with applications that use green
threads as small code changes may result in radically different
schedules.

Thus, there are some types of changes for which {\dora} will not find the
$d$-optimal replay.  Fortunately, {\dora} produces enough information for the
user to identify when these conditions occur. This allows the user to
distinguish behavior caused by an application change from behavior due to
{\dora}'s limitations and makes {\dora} a useful tool for debugging and
validation.

These conditions may seem restrictive, but they are often not an issue in
practice because patches rarely change core
application semantics. In a study of 60 patches each for MySQL, Apache, OpenSSL,
and Squid, 83\% resulted in only minor changes to the application behavior.
Analysis of various security patches showed that over 75\% only
changed applications in minor ways ~\cite{mreplay-feas}. This makes sense, as
patches often attempt to fix an edge case in an application.
\section{Evaluation}
\label{dora:sec:results}


\begin{sidewaystable}
\begin{tabular}{|l|l|l|l|} \hline
{\bf Name}     & {\bf Description}             & {\bf Problem/Exploit Workload}        & {\bf Production Workload}             \\ \hline
		apache-log & Apache 2.4.2 web server       & Log format change (Apache Bug 53131)  & {\tt httperf 0.8} with 100KB web page \\ \hline
		apache-sec & Apache 2.2.19 web server      & DoS attack (CVE 2011-3192)            & {\tt httperf 0.8} with 100KB web page \\ \hline
		exim       & Exim 4.69 mail server         & Privilege escalation (CVE 2010-4344)  & Send 1000 1KB e-mail messages         \\ \hline
		mysql      & MySQL 5.0.67 database server  & Unauthorized access (CVE 2008-2079)   & {\tt sql-bench}                       \\ \hline
		nginx      & Nginx 0.8.14 web server       & Crash server (CVE 2009-2629)          & {\tt httperf 0.8} with 100KB web page \\ \hline
		proftpd    & ProFTPD 1.3.0 ftp server      & Crash server (CVE 2006-5815)          & 100 clients fetch 10MB file           \\ \hline
		redis      & Redis 2.4.11 key-value store  & Request with insufficient logging     & {\tt redis-benchmark} with 50 clients \\ \hline
		squid      & Squid 3.1.7 http proxy server & DoS attack (CVE 2010-3072)            & {\tt ab 2.3} with cached facebook.com \\ \hline
		wget       & wget 1.11.4 http client       & Create arbitrary file (CVE 2010-2252) & 100 requests to http://www.cnn.com    \\ \hline
\end{tabular}
\caption{Application scenarios}
\label{dora:tab:scenarios}

\vspace{1em}

\begin{tabular}{|l|l|l|}   \hline
{\bf Name}     & {\bf Debugging Change}                          & {\bf Replay Mutations}                                                           \\ \hline
		apache-log & Modify log format in configuration file         & Add 1 {\tt fstat()}, 1 {\tt mmap()}, and 3 {\tt write()} per request             \\ \hline
		apache-sec & Add print statements for debugging              & Add 1 {\tt fstat()}, 1 {\tt mmap()} and then 2 {\tt write()} per request         \\ \hline
		exim       & Recompile with debugging options enabled        & None                                                                             \\ \hline
		mysql      & Add conditional print statements for debugging  & Add 1 {\tt fstat()}, 1 {\tt mmap()}, 1 memory event, 12 {\tt write()}            \\ \hline
		nginx      & Change the config file to enable debug messages & Add 508 {\tt write()}, delete 1064 syscalls                                      \\ \hline
		proftpd    & Recompile with debugging options enabled        & Add 1 {\tt close()}                                                              \\ \hline
		redis      & Log erroneous client requests                   & Add 1 {\tt open()}, at least 3 {\tt write()} and 1 {\tt close()} per request     \\ \hline
		squid      & Save parsed requests to file                    & Add 1 {\tt open()}, at least 10 {\tt write()} and 1 {\tt close()} per request    \\ \hline
		wget       & Change language from Italian to Japanese        & Replace 17 {\tt write()}, 24 {\tt mmap()}, 2 {\tt open()}, and delete 5 syscalls \\ \hline
\end{tabular}
\caption{Application modifications for debugging}
\label{dora:tab:debugging}

\end{sidewaystable}

We have implemented a prototype of {\dora} in Linux. The recorder and
replayer run in kernel space while the explorer runs in user space.
Although the prototype only instruments a subset of the Linux kernel API, we
demonstrated the functionality of this prototype in diagnosing
and fixing bugs and measured its performance overhead with nine
widely used real-world, multi-process, and multi-threaded applications
and 32 different application changes involving thousands of lines of code.
For our experiments, we used version 2.6.35 of the Linux kernel,
Python 2.6.6, Cython 0.14, and UnionFs-Fuse 0.23.
Measurements were done on a set of HP DL360 G3 servers, each with dual
3.06~GHz Intel Xeon CPUs, 4~GB RAM, and dual 18~GB local disks.

We recorded a wide
range of applications with various workloads that exhibited bugs,
as listed in the third
column of Table~\ref{dora:tab:scenarios}.
We verified that {\dora} can
deterministically replay the original recorded applications and then
replayed the executions with modified applications.
Section~\ref{dora:sec:debugging} shows that
{\dora} can replay these workloads using reconfigured or
modified applications with additional debugging or other
instrumentation. Section~\ref{dora:sec:validation} shows
how {\dora} can replay the exploits in Table~\ref{dora:tab:scenarios}
using patched versions of the applications to help developers
verify that the bug patches successfully resolve the problems.
It also shows how {\dora} can
replay the workloads listed in the last column of
Table~\ref{dora:tab:scenarios} using the patched versions of the
applications to help system administrators verify that the patches
do not introduce errors in production workloads.
Section~\ref{dora:sec:upgrades} shows that {\dora} can
verify production workloads on a series of release upgrades for the
applications listed in Table~\ref{dora:tab:upgrades}.
Finally, Section~\ref{dora:sec:performance} presents record-replay
overhead for the production workloads.

\begin{table}[t]
\begin{center}
\small
\begin{tabular}{|l|l|l|}   \hline
  {\bf Name} & apache-upgrade & redis-upgrade \\
\hline\hline
{\bf Start Version} & Apache 2.2.19 & Redis 2.4.1 \\
\hline
{\bf Upgrades} & 3 (2.2.20 - 2.2.22) & 12 (2.4.2 - 2.4.13) \\
\hline
{\bf Commits} & 277 & 137 \\
\hline
{\bf LOC} & 5179+, 388- & 2942+, 1154- \\
\hline
{\bf Workload} & {\tt httperf 0.8} & {\tt redis-benchmark} \\

\hline
\end{tabular}
\end{center}
\renewcommand\thetable{1}
\caption{Application upgrades}
\label{dora:tab:upgrades}
\end{table}

\subsection{Debugging and Diagnosis Techniques}
\label{dora:sec:debugging}

We used a wide range of debugging and diagnosis techniques with the
workloads listed in the third column of Table~\ref{dora:tab:scenarios}.
Since our work focuses on diagnosing and fixing bugs,
most of the problems involve known security vulnerabilities,
as indicated by their Common Vulnerabilities and Exposures (CVE) identifiers.
However, the apache-log scenario shows a previously unknown bug in Apache
that we found with {\dora}, and the Redis scenario shows how to add
retroactive logging without discussing a specific bug.

For each of these exploits, we show how {\dora} can be used to diagnose
the cause of a bug. We consider debugging techniques an experienced
developer might apply to identify the root cause of each problem.
Table~\ref{dora:tab:debugging} lists the application changes needed to use various
debugging and diagnosis techniques for each scenario.
{\dora} successfully found the $d$-optimal replay for all of these
application changes. Table~\ref{dora:tab:debugging} shows the needed replay mutations.

{\bf apache-log} was originally intended to show how {\dora} could be used
for retroactive logging, but ended up
showing {\dora} finding a previously unknown Apache bug~\cite{apache-bug-53131}.
We wanted to use {\dora} to add user agent and referrer information to Apache
log files, since these could provide useful usage statistics for a website
administrator. Although the default Apache logging configuration will not log
this information, {\dora} records all HTTP
header information that the server receives. Thus, an administrator using {\dora}
could modify a configuration file to include this information
and replay the recorded execution with the modified configuration
to generate the desired web server log.

However, doing
this yields a log file with incorrectly truncated entries. This
behavior was due to a
previously unknown bug in Apache.
In several places in code, Apache mistakenly assumes that a call
to \code{write()} will either write the desired amount of bytes or
fail, instead of checking the return
value and calling {\tt write()} until all the required bytes are
written. We submitted a bug report and patch to Apache~\cite{apache-bug-53131} which was
accepted into the codebase.

{\bf apache-sec} records an exploit of a heap overflow
vulnerability that launches a denial of service attack against an Apache
web server using only a handful of requests. By examining Apache's log,
an experienced Apache developer will notice oddities in some of the requests,
but will not have enough information to identify the bug with the default
logging settings. In particular, it would be helpful to have more
information about the range headers of the requests. Using {\dora}, a
developer can add print statements to Apache, replay, and recognize that the
problem was due to incorrect handling of overlapping range headers. Five system
calls were added for each request as a result of the additional print
statements.

{\bf exim} involves an exploit that crashes the mail server using a
heap overflow vulnerability in a buggy string formatting function
that allows attackers to execute arbitrary code. If this crash was recorded in
the wild, it would be helpful to use GDB to analyze the program at the time of
the crash. However, production servers are almost always optimized and compiled
without debugging symbols. Thus, a traditional record-replay system would be
unable to help. Using {\dora}, a developer can recompile the
program, replay the exploit
using the recompiled program, and hook GDB to the replayed program
before it crashes. When investigating the stack trace, the nature of
this attack becomes clear. No mutations were needed in the $d$-optimal replay,
despite various memory layout changes to the program as a result of
recompilation.

{\bf mysql} involves an exploit which maliciously uses symlinks
to elevate permissions to a database. By default, MySQL disables
logging. Thus, a developer trying to discover how
a malicious user gained access to a database
will have no information about which commands
were executed. Using {\dora}, the developer can modify the program
to log executed commands, then replay the exploit
using the modified program. This process can be repeated, allowing
the developer to iteratively add print statements to different sections of the
code and pinpoint the bug. To demonstrate this, we added enough print statements
to identify the bug. {\dora} found the $d$-optimal replay, which had
mutations of fourteen added system calls and a shared memory event.

{\bf nginx} involves running Nginx, a high-performance HTTP server,
and crashing one of its worker processes with a malicious HTTP request
that uses a buffer underflow attack to execute arbitrary code.
The default log does not show what actions were taken on
each request, which makes debugging difficult. Using {\dora}, the
developer can modify the configuration file to enable verbose logging and
replay the exploit with the modified configuration.
To generate verbose logs on a workload exhibiting the exploit, 508 {\tt write()}
calls were added and 1064 system calls were deleted.

{\bf proftpd} records an exploit that crashes the FTP server by taking
advantage of an off-by-one error to execute arbitrary code. As with the exim
use case, GDB would be a helpful debugging tool, but a production server is
unlikely to be compiled with debugging symbols. With {\dora}, a developer can
recompile
with the Makefile configuration for debugging, replay
the exploit using the recompiled program, and hook
GDB to the replayed program before it crashes. The resulting stack
trace makes it easy to diagnose the problem. A replay mutation of
adding 1 {\tt close()} was needed to use the debugging configuration.

{\bf redis} involves recording Redis, an in-memory key-value store often
used in production applications as a caching layer on top of a general purpose
database. This use case does not involve a specific bug but instead shows how a
developer can add logging to Redis and replay this modification on the original
recording, effectively turning on retroactive logging. Replay mutations of
adding at least
five system calls per request was needed. The number of additions varied
based on the nature of the request. For instance, a malformed request triggered
more logging than a proper request.

{\bf squid} involves an exploit that crashes the Squid daemon by sending an
empty Expect HTTP header parameter.
The default request logging does not provide enough information about the header to
determine the cause of the bug. Using {\dora}, a developer can
modify the program to log each request to a file with the complete
header information of the request, then replay a recording of the exploit with
the modified program. This allows the developer to see that the
requests which crash the server have empty header parameters and
narrow down the bug to a very specific section of code.
At least 12 system calls were added per request. The exact number varied
depending on the type of request.

{\bf wget} involves downloading a file from a malicious server
that does a 301 redirect. A vulnerability in wget allows the
server to choose the destination filename.
Remote servers can create or overwrite arbitrary files and even
execute arbitrary code by writing
dotfile in a home directory. Because this behavior depends on a live server
behaving in a particular way, this issue may be difficult to reproduce and debug
if it is not noticed immediately. Additionally, to
demonstrate the robustness of {\dora}, we suppose that an Italian developer
observes this behavior and wants to show it to a Japanese developer
who cannot reproduce the results because
the server is no longer available. Using {\dora}, the second
developer can replay wget in a different language, enabling
collaborative debugging across international borders and language
barriers. While this scenario is tongue-in-cheek,
the translation use case is novel and has interesting
applications. The replay involved replacing 43 system calls and deleting 5
system calls.

\subsection{Patch Validation}
\label{dora:sec:validation}


\begin{sidewaystable}
% \begin{table*}[t]
\begin{center}
\begin{tabular}{|l|l|l|}   \hline
{\bf Name}     & {\bf Patch LOC +/-} & {\bf Replay Mutations}                                                                           \\ \hline
		apache-log & 39+, 39-            & Add 1 {\tt write()} per truncated log entry                                                      \\ \hline
		apache-sec & 292+, 154-          & Add 1 {\tt write()} and replace 1 {\tt writev()} per request                                     \\ \hline
		exim       & 7+, 0-              & Delete 18 syscalls, add 29 syscalls                                                              \\ \hline
		mysql      & 170+, 60-           & Add 29 {\tt lstat()}, delete 79 syscalls, add 1 {\tt write()}, delete 15 and add 8 memory events \\ \hline
    nginx      & 9+, 5-              & Delete 1 {\tt write()} and replace 1 {\tt writev()} per request, then go live on crash           \\ \hline
		proftpd    & 17+, 3-             & Delete 694 syscalls, add 38 syscalls, delete a {\tt SIGSEGV}                                     \\ \hline
		squid      & 38+, 33-            & Delete a {\tt SIGSEGV}, 1 {\tt close()}, 1 {\tt stat()}, 1 {\tt write()}, then go live           \\ \hline
		wget       & 43+, 12-            & Replace 9 syscalls among {\tt stat()}, {\tt write()}, {\tt open()}, {\tt utime()}                \\ \hline
\end{tabular}
\end{center}
% \renewcommand\thetable{5}
\caption{Application patches tested against exploits}
\label{dora:tab:exploits}
% \end{table*}
\end{sidewaystable}

For each bug exhibited by the workloads listed in the third column of
Table~\ref{dora:tab:scenarios}, we replayed the
bug-inducing workload on the patched application to verify
that the patch successfully fixed the bug. Table~\ref{dora:tab:exploits}
lists the number of lines of code added and deleted for each patch and the
mutations needed for each replay.
Redis is not included; since it did not involve an application bug, no
patch was necessary. Table~\ref{dora:tab:exploits} shows three interesting points.

First, {\dora} found the $d$-optimal replay
even with substantial patches of
over 400 lines of code changed. The replay mutations
that were needed to find the $d$-optimal replay varied. Many involved
executing different system calls, but others involved changes in
signal delivery and shared memory accesses. This demonstrates
{\dora}'s ability to replay despite a broad range of application
modifications so long as the core application semantics remain the
same.

Second, we show that production workloads can be replayed using patched
applications to verify that the patch does not introduce errors
into those workloads.
For each workload listed in the last column of Table~\ref{dora:tab:scenarios},
{\dora} recorded the workload using the unpatched application, then found a
$d$-optimal mutable replay using patched versions of each application.  We
examined the output of each mutable replay and verified that the patches did not
change application behavior when running the workload.  We also compared each
original recorded log with the log of the corresponding mutable replay to verify
that the patches did not change the application execution in unexpected ways.
System administrators could use this technique to test patches before deploying
them to have more confidence that they will not break their production systems.

Third, Table~\ref{dora:tab:exploits} shows that the go live feature of the replayer
can be used to validate patches even when a recorded exploit crashes a process.
The exploit for Nginx caused a worker to crash, and the Squid exploit crashed the
entire application.
In both cases, {\dora} does not replay
the original {\tt SIGSEGV} and allows the applications to go live
and handle new requests. Although a worker process crashed in the proftpd
scenario, {\dora} did not go live because the
proftpd master forks a new worker per connection and is resilient to worker
crashes, allowing subsequent requests to be replayed without going live.

\subsection{Release Upgrades}
\label{dora:sec:upgrades}

To demonstrate another use case of {\dora}, we took two server applications
and recorded them running the benchmarks we used as production
workloads as listed in Table~\ref{dora:tab:upgrades}. We then replayed
those executions over a series of 15 release upgrades to verify that
the workloads continued to function correctly across upgrades. {\dora} found
the $d$-optimal replay in all of these cases.

{\bf apache-upgrade} consists of a series of upgrades of Apache over
an 8 month timeframe from 2.2.19 (May 21, 2011) to 2.2.22 (Jan 30,
2012). The upgrades involved changes of more than 5000 lines of code 
and 277 separate commits. Using {\dora}, we recorded version
2.2.19 running httperf, then replayed the recording with each subsequent
version. We repeated this process for versions 2.2.20 and 2.2.21.
{\dora} successfully replayed all of these
application changes. This required various add and delete mutations of
{\tt read()} and {\tt brk()} calls.  Note that we also tried
this experiment starting with Apache 2.2.18, but {\dora} was unable
to replay from that version due to core library modifications
that caused large shared memory layout changes between Apache 2.2.18
and 2.2.19.

{\bf redis-upgrade} consists of a series of upgrades of Redis over a
7 month timeframe from 2.4.1 (October 17, 2011) to 2.4.13 (May 2,
2012).  The upgrades involved changes of more than 4000 lines of code
in 137 separate commits.  Using {\dora}, we recorded version 2.4.1
running redis-benchmark, then replayed the recording with each later
version. We also repeated this experiment for version 2.4.2 and
upgrades 2.4.3 to 2.4.13, version 2.4.3 and upgrades 2.4.13 to 2.4.4, and so on.
{\dora} successfully replayed all of these
application changes. They required add and delete mutations of
12 different system calls, including {\tt open()}, {\tt close()}, {\tt
read()}, {\tt write()}, {\tt mmap()}, {\tt munmap()} and {\tt time()}
system calls.

\subsection{Performance}
\label{dora:sec:performance}

To quantify the performance costs of using {\dora}, we measured the
runtime overhead of recording and replaying the production workloads 
listed in the last column of Table~\ref{dora:tab:scenarios}.
Table~\ref{dora:tab:performance} shows the overhead of recording the
production workload with the unpatched application,
the storage growth rate of recording, and the
speedup when replaying the recording with the patched application.
Unless otherwise noted, default configuration options were used for
all applications.  The standard deviations for all measurements were
negligible.

\begin{table}[t]
\begin{center}
\begin{tabular}{|l|l|l|l|}   \hline
  {\bf Name}   & {\bf Recording } & {\bf Storage } & {\bf Replay } \\
               & {\bf Overhead}   & {\bf Growth}   & {\bf Speedup} \\ \hline
		apache-log & 9.3\%            & 31 KB/s        & 3.8x          \\ \hline
		apache-sec & 4.8\%            & 18 KB/s        & 1.9x          \\ \hline
		exim       & 4.3\%            & 30 KB/s        & 7.2x          \\ \hline
		mysql      & 4.7\%            & 9.6 KB/s       & 1.1x          \\ \hline
		nginx      & 9.7\%            & 15 KB/s        & 2.2x          \\ \hline
		redis      & 2.6\%            & 91 KB/s        & 1.3x          \\ \hline
		proftpd    & 4.1\%            & 22 KB/s        & 2.6x          \\ \hline
		squid      & 8.2\%            & 124 KB/s       & 1.2x          \\ \hline
		wget       & 2.2\%            & 19 KB/s        & 11x           \\ \hline
\end{tabular}
\end{center}
\renewcommand\thetable{4}
\caption{Mutable replay performance}
\label{dora:tab:performance}
\end{table}

The recording overhead in all cases was less than
10\% even for CPU-bound workloads designed to stress application
performance. For example, Squid performance was measured with a fully
cached web page, resulting in a CPU
intensive workload. Even with these unfavorable workloads, the results indicate
that {\dora} can be used in production systems with modest overhead.

Similarly, the time to replay the original recording on the original application
was in all cases faster than the original execution; in one case, it was over an
order of magnitude faster.
This is because {\dora} can bypass blocking system calls that sleep.
Since production servers are likely to sleep more and
service requests less frequently than in our benchmarks, replay
speedup will be much higher in practice.

While recording, the log was streamed through {\tt gzip} before being persisted to
disk. The storage growth rates ranged from 10 KB/s to 130 KB/s. These
storage requirements are modest considering our workloads. {\dora}
would take almost three months to fill a 1 TB drive at the worst
of these rates, which makes it an affordable and practical solution.
\section{Related Work}
\label{dora:sec:related}

Many record-replay approaches have been proposed to improve bug 
reproducibility
debugging~\cite{idna:vee06,decouple:usenix08,instant-replay,r2:osdi,odr:sosp09,pinsel:pldi07,pres:sosp09,jockey,srinivasan:flashback,subhraveti:sigmetrics11},
but none allows for mutable replay. Some
approaches
propose relaxing the requirement of deterministic replay for performance
reasons. For example, ODR~\cite{odr:sosp09} proposes only ensuring that the
output is deterministically replayed for replay debugging. This is
quite different from mutable replay, in which the output may
change due to application changes.

Some record-replay systems can support a form of deterministic replay
that may differ in limited ways from the original recorded
execution. Crosscut~\cite{crosscut} can reduce the information
recorded in a log
so that, for example, sensitive information can be
purged before
replay. Our previous work on Scribe~\cite{scribe:sigmetrics10} replays a recorded
application execution until a specified
point, and then transitions to
live execution instead of replaying the rest of the
log. Our previous work on Racepro~\cite{racepro:sosp11} detects process races due to
dependencies in the ordering of system calls by recording an
application execution to a log, identifying a pair of system calls
that may be racy, truncating the log at the occurrence of the pair of
system calls, inverting their order, and then replaying the truncated
log with the reordered system calls to detect process
races.  However, Racepro only supports changes that reorder system calls and
does not support changes in the middle of replay.
None of these approaches supports mutable replay,
but mutable replay could be useful for some of these systems.  For
example, Racepro could use mutable replay to avoid replay
divergence and more effectively detect process races.  Another race
detection tool~\cite{pinsel:pldi07} uses the iDNA~\cite{idna:vee06}
record-replay framework and would also benefit from mutable replay.

A few record-replay systems allow new code to be run while replaying a recorded
execution~\cite{intrusions:sosp05,decouple:usenix08}. However, this new code
cannot have any side effects on the program.  If a replay diverges due to new
code, these systems must rollback to a point prior to the divergence for the
replay to continue.  In contrast, {\dora} allows replay to continue even
after divergence; side effects due to new code are preserved.  Moreover, unlike
{\dora}, these other approaches prevent application developers from leveraging existing
configurable application functionality and instead require that developers learn
a new complex system. Because these other approaches work at a VM level, they are
fundamentally limited in their abilities to perform mutable replay and support
the kind of application changes supported by {\dora}. Finally, none of these
other approaches work on multicore or multiprocessor systems.

A concept of mutable replay was mentioned as a part of
DSF~\cite{dsf}, a Java-only framework for implementing distributed 
algorithms. DSF recognized that existing replay approaches did not
allow adding print statements for debugging.  DSF requires that all
applications to be written using its framework, requires modification
to the applications, and is primarily simulation-based. Furthermore, DSF
presents no algorithms or mechanisms for actually doing mutable replay, and
presents no experimental results demonstrating the ability to do mutable
replay.  More recently, a study has assessed the potential utility of
mutable replay on real patches~\cite{mreplay-feas}, though no mutable
replay system or results are presented.
{\dora} presents the first system that achieves transparent mutable replay,
requires no application
modifications, and demonstrates experimentally that
mutable replay can be used with real applications.

Alternative techniques have been proposed to help with patch validation, one
use case of mutable replay.
Band-aid
patching~\cite{bandaid} and delta execution~\cite{delta} instrument patches to
identify portions of an application that have changed, execute both
the unpatched and patched code paths either serially or in parallel,
and select the results from one path or merge the results from both
paths. However, these approaches incur substantial performance overhead.
Many simple patches cannot be handled by these
approaches, such as simple changes to data structures. Unlike {\dora},
these approaches do not allow patch validation on a recorded bug or offline
patch validation on a production workload.
Furthermore, they are designed only for patch validation and are not effective for
debugging.

Self-healing systems have been proposed which 
record the occurrence of a bug, then automatically generate and apply
a patch as a temporary fix to the problem~\cite{assure:asplos09}.
{\dora} is complementary
to these systems and can be used to
verify that a generated patch successfully fixes
problems that occurred in the original workload.

Finding a mutable replay has some similarities
to the edit distance and longest common subsequence
problems, which have applications to approximate string matching and
bioinformatics. In those problems however, both
sequences being used for matching are known in advance. In contrast,
mutable replay must match a known execution log with an
execution sequence that is not known in advance, so these algorithms cannot be
directly applied.
\section{Conclusions and Future Work}
\label{dora:sec:conclude}

{\dora} introduces the concept of mutable record-replay and is the
first transparent mutable record-replay system.  It enables, for the first time,
a recording of an application execution to be replayed
using a modified version of the application for a large class of
application changes.  This is made possible by the use of lightweight
operating system mechanisms to record and replay without imposing
unnecessary timing and ordering constraints.  {\dora} introduces an
explorer that directs the replay mechanism to identify a mutable
replay of the modified aplication that minimizes differences with the
original unmodified application execution.

Our experimental results on a Linux prototype demonstrate that mutable
replay is feasible across a wide range of real-world applications and
application changes which can reach thousands of lines of code, even
without support for major changes to core application semantics.  We
show that mutable replay is useful for enabling common debugging
techniques not possible with previous record-replay systems.  We
also show that mutable replay enables validation of security patches
against both exploits and production workloads. This is all
accomplished without requiring source code modifications and with low
recording overhead, enabling usage on production systems.  These
results demonstrate that mutable replay has the potential to enable
new techniques for debugging and patch testing and validation, which
can lead to substantial improvements in software reliability and developer
productivity.

We hope to explore a number of directions in future work.
While we have explored a few ways in which mutable replay can be used in
multicore debugging and patch validation, {\dora} provides a
foundation for exploring other uses of mutable replay as well.
We have evaluated a particular cost function and exploration
algorithm for mutable replay, but much more can be done to consider
alternatives that may be
particularly suited for other use cases.  Furthermore, it would be
beneficial to perform instrumentation at a higher level:
if runtime systems like the Java
Virtual Machine or the Ruby MRI were instrumented to work with
{\dora}, mutable replay could be an even more effective tool to
reproduce, diagnose, and fix software bugs for Web and mobile
applications.
